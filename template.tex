%FILL THESE OUT HERE
%\\\\\\\\\\\\\\\\\\\\\\\\\\\\\\\\\\\\\\\\\\\\\\\\\\\
\def \assignmentname {Place Assignment Here}
\def \thisclass {Place Course Here}
%//////////////////////////////////////////////////
\documentclass{article}
\usepackage{amsmath}
\usepackage{bm}
\usepackage{amssymb}
\usepackage{enumerate}
\usepackage{setspace}
\usepackage{fancyhdr}
\usepackage{algpseudocode}
\usepackage{listings}

\usepackage[mathscr]{euscript}
\usepackage[left=3cm,right=3cm,top=4cm,bottom=4cm]{geometry} %smaller margins!
\usepackage{hyperref}
\hypersetup{
    colorlinks=true,
    linkcolor=blue,
    filecolor=magenta,      
    urlcolor=cyan,
}
\usepackage{graphicx}
\graphicspath{ {.} }
\usepackage{tikzlings}

\newcommand{\N}{\mathbb{N}}
\newcommand{\Z}{\mathbb{Z}}
\newcommand{\R}{\mathbb{R}}
\newcommand{\C}{\mathbb{C}}
\newcommand{\Q}{\mathbb{Q}}
\newcommand{\I}{\mathbb{I}}
\newcommand{\U}{\mathcal{U}}
\newcommand{\E}[1]{\mathbb{E} \left[ #1 \right] }
\renewcommand{\P}[1]{\mathbf{P} \, \{ #1 \} }
\newcommand{\Var}[1]{\mathsf{Var} ( #1 )}

\newcommand{\st}{\text{ such that }}
\newcommand{\wlogt}{\text {without loss of generality}}

\newcommand{\wqed}{\hfill $\square$} %Quite Easily Done square
\newcommand{\question}[1]{\section*{Question \##1}}
\newcommand{\proof}{\subsection*{Proof}}

\pagestyle{fancy}
\fancyhf{}
\rhead{\assignmentname}
\lhead{\thisclass}
\lfoot{Dear TA grading this, I hope you have a nice day :) \href{}{PSET-related meme}}
\rfoot{Page \thepage}

%define info for the document
\begin{titlepage}
    \title{\textbf{\assignmentname}}
    \date{\today}
    \author{Sachit Lumba \thanks{Groupmates: General: Zack Sussman}}
\end{titlepage}

\onehalfspacing

\begin{document}
\maketitle

\newpage
\question{1}
How many licks does it take to get to the center of a tootsie pop?

\proof
Here goes your proof! You can use math symbols like this $\forall x \in \R, x = x$.

\begin{itemize}
    \item Using this environment, we can add bullet points.
    \item Like this.
\end{itemize}

\begin{enumerate}
    \item Using this environment, we can number lists.
    \item Like this.
\end{enumerate}

\begin{itemize}
    \item[\textit{One:}] Using brackets next to our slash item, we can customize symbol is used in a list.
    \item[\textit{Two:}] Like this!
\end{itemize}

If we want to break a line, we can use the newline command \newline and Latex gives us a newline. \newline 
To write formulas on their own, isolated line, use double dollar signs. For example: $$\forall x \in \R, x \neq x+1$$
And for the advanced users, if you want to align multiple lines of math mode text along a common symbol (like the $=$ symbol), you use the align environment with \&.

\begin{align*}
    0 &= x^2+2+1 \\ 
    y+1 &= x+1 \\ 
    3.1415926535 &\approx \pi \\
\end{align*}

\tikz\pig; Contradiction pig

\end{document}



